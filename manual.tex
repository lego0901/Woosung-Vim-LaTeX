
\documentclass[a4paper, 10pt]{article}

\usepackage{woosung_hw}
\usepackage{woosung_korean}
\usepackage{woosung_math}

\title{Vim 에서의 LaTeX 환경 설정 및 사용법}
\author{}
\date{}

\newcommand{\bs}{\textbackslash}

\begin{document}

\maketitle

\section{들어가며}
제가 사용하는 LaTeX 환경설정에 대한 파일입니다. 엄청 간단하고 미니멀합니다.
그런데 작업은 무지 빨라지고 편해집니다. 공유를 하면 참 좋겠다고 평소에 생각을 많이
했었습니다. \texttt{.vimrc} 설정이나, \texttt{.sty} 파일 등이 너무 배배 꼬여서 보여주기
부끄러운 상태라 여태껏 공개를 안 했습니다. 이번 여유가 있을 때 정리를 해서 메뉴얼까지
배포하려고 합니다. 부족한 글 솜씨지만 넓은 아량으로 이해해주시면 감사하겠습니다.


\subsection*{제작자 컴퓨터 환경}
저는 \textbf{Ubuntu 16.04 LTS (Gnome)} 환경에서 작업을 합니다. 컴파일은
\texttt{pdflatex} 명령어를 통하여 합니다. \textbf{메뉴얼의 모든 작업은 리눅스 환경에서
이루어집니다.} LaTeX은 터미널에 다음과 같은 명령어를 입력함으로써 설치하였습니다.
\begin{Verbatim}[tabsize=4,xleftmargin=2em]
$ sudo apt-get update
$ sudo apt-get install texlive-full
\end{Verbatim}

윈도우에서는 운영체제에 호환되는 Vim과 texlive의 설치보다는 리눅스 에뮬레이터인
\textbf{bash}를 이용해 설치를 하시기를 권장드립니다.



\section{Vim 설정}
설치는 간단하게 넘어가도록 하겠습니다. Ubuntu 환경에서는 다음과 같이 설치할 수 있습니다.
다음 코드를 터미널에 입력하시면 됩니다.
\begin{Verbatim}[tabsize=4,xleftmargin=2em]
$ sudo apt-get update
$ sudo apt-get install vim
\end{Verbatim}

우분투의 경우 \texttt{vim}이 아닌 \texttt{vim-tiny} 버전으로 이미 설치가 되어있을 가능성이
있습니다. 이러할 경우
\begin{Verbatim}[tabsize=4,xleftmargin=2em]
$ sudo apt-get purge
$ sudo apt-get install vim
\end{Verbatim}
으로 재설치를 하시길 바랍니다.

터미널에서 \texttt{vim} 이라고 입력하신 뒤 엔터를 누르시면 바로 실행시킬 수 있습니다.

Vim 에서는 여러 가지 플러그인들을 깔면 매우 편합니다. LaTeX 문서를 작업할 때는 특히 
\begin{itemize}
	\item YouCompleteMe (리눅스, OS X 환경에서만 설치 권장)
\end{itemize}
를 추천드립니다. Latex-Suite 등 다른 유명한 플러그인들도 있다만, 한글로
문서를 작업할 때는 영 좋지 못한 것 같습니다. 그래서 LaTeX에 직접적으로 연관된
플러그인들의 설치는 지양하도록 합시다.

플러그인 설치 및 \texttt{.vimrc} 파일 설정 방법은
\href{https://github.com/lego0901/My-Vim-Settings}{Github 포스팅}을 참고하시길 바랍니다.


\subsection*{.latexrc 설정}
Vim으로 LaTeX을 쓰는데 편리한 단축키 소스들을 \texttt{.latexrc} 으로 작성하여 첨부하였습니다.
이를 \\ \texttt{\$HOME/.vim/.latexrc} 위치에 집어넣고 아래 코드를 \texttt{.vimrc} 파일의
맨 아래쪽에 집어넣어 줍시다.
\begin{Verbatim}[tabsize=4,xleftmargin=2em]
source $HOME/.vim/.latexrc
au BufRead *.latexrc set filetype=vim
\end{Verbatim}

\texttt{.latexrc} 파일에 있는 내용은 메뉴얼의 뒷 부분에서 다 설명드리도록 하겠습니다.
모두 다 결국 \texttt{nmap}과 \texttt{imap}으로만 이루어진 간단한 코드임을 확인하실
수 있을 것입니다. 각자 자주 사용하는 방식대로 임의로 수정하셔도 괜찮습니다.



\section{스타일 파일 설정}
LaTeX에서 \texttt{sty} 파일은 C의 헤더파일, 자바나 파이썬의 다른 클래스 파일과 비슷하다고
보시면 됩니다. 예를 들어 보겠습니다. \texttt{woosung.sty} 파일에 다음과 같이 입력하여 
저장했다고 합시다.
\begin{Verbatim}[tabsize=4,xleftmargin=2em]
\usepackage{setspace}
\usepackage{kotex}
\newcommand{\hw}{Hello World!}
\end{Verbatim}
같은 폴더 안에 \texttt{tex} 파일을 만들어 다음과 같이 입력하고 컴파일하면
\begin{Verbatim}[tabsize=4,xleftmargin=2em]
\documentclass{article}
\usepackage{woosung}
\begin{document}
	\setstretch{1.25}
	안녕하세요! 한글이 입력이 되네! \hw
\end{document}
\end{Verbatim}
아마 다음과 같은 내용의 \texttt{pdf} 문서가 나올 것입니다.
\begin{Verbatim}[tabsize=4,xleftmargin=2em]
안녕하세요! 한글이 입력이 되네! Hello World!
\end{Verbatim}
원래 \Verb+\setstretch+ 명령어를 사용하거나, 한글 문서를 조판하려면 \texttt{setspace},
\texttt{kotex} 패키지를 각각 추가해줘야합니다. \Verb+\hw+ 명령어로 \texttt{Hello World!}
를 표현하기 위해서는 \Verb+\newcommand{\hw}{Hello World!}+ 코드가 필요하고 말이죠.
스타일 파일에 이 패키지를 불러오는 코드를 적고, 이 스타일 파일을 \Verb+\usepackage+
명령어로 불러옴으로써 함께 로드가 된 것입니다. 이와 같이 자주 사용하게될 긴 명령어들은
\texttt{sty} 파일로 만들어 미리 저장해두는 것을 권장합니다.


\subsection*{전역적으로 접근할 수 있게 만들기}
매우 자주 쓰는 \texttt{sty} 파일들은 어디에서나 접근할 수 있도록 따로 처리해줄 수
있습니다. 터미널에서 다음과 같은 명령어를 입력하세요.
\begin{Verbatim}[tabsize=4,xleftmargin=2em]
$ cd $HOME && mkdir .texmf && mkdir .texmf/tex && mkdir .texmf/tex/latex
$ kpsewhich texmf.cnf
\end{Verbatim}
두 번째 줄의 명령어로 나온 경로에 있는 파일을 아래 명령어로 여세요.
가령 \Verb+/etc/texmf/web2c/texmf.cnf+ 로 나왔다고 하면, 이렇게 입력하시면 됩니다.
\begin{Verbatim}[tabsize=4,xleftmargin=2em]
$ gedit /etc/texmf/web2c/texmf.cnf
\end{Verbatim}
파일이 열리면 \Verb+TEXMFHOME+ 이라는 항목을 찾아서 이렇게 바꿔줍니다.
\begin{Verbatim}[tabsize=4,xleftmargin=2em]
$ TEXMFHOME = ~/.texmf
\end{Verbatim}
이제 \texttt{\$HOME/.texmf/tex/latex} 폴더 안에 자주 쓰는 \texttt{sty} 파일을 집어넣으세요.
그런 뒤, 다시 터미널로 돌아와서 아래와 같은 명령어를 실행시켜주세요.
\begin{Verbatim}[tabsize=4,xleftmargin=2em]
$ sudo texhash
\end{Verbatim}
이제부터 따로 해당 \texttt{sty} 파일을 같은 디렉토리에 넣을 필요 없이 어디서든 접근
가능하게 됩니다.


\subsection*{Woosung LaTeX 스타일 파일}
첨부된 \texttt{woosung\_hw.sty}, \texttt{woosung\_korean.sty}, \texttt{woosung\_math.sty}
파일은 제가 유용하게 사용하는 스타일 파일들입니다. 이것도 매우 미니멀합니다.
\begin{itemize}
	\item \texttt{woosung\_hw.sty}: 숙제 파일을 만들 때 주로 사용하던 양식인데,
		실험 리포트를 비롯한 대부분의 문서에서 이 스타일을 불러와 사용합니다.
	\item \texttt{woosung\_korean.sty}: 한글로 문서를 조판할 때 불러와 사용합니다.
		\Verb+\setstretch+ 로 문장간 간격에 대한 조정이 되어있습니다.
	\item \texttt{woosung\_math.sty}: 수식을 입력할 때 매우 편하게 사용하기 위해
		만들었습니다. Theorem, Definition 등을 입력해야할 경우 유용합니다.
\end{itemize}

특히, \texttt{woosung\_math.sty} 파일에 있는 command들이 매우 편리합니다.
파일을 열어보면 다음과 같은 코드가 있습니다.
\begin{Verbatim}[tabsize=4,xleftmargin=2em]
...
\newcommand{\R}{\mathbb R}
\newcommand{\N}{\mathbb N}
\newcommand{\Z}{\mathbb Z}
% Analysis
\newcommand{\ep}{\epsilon}
\newcommand{\dt}{\delta}
\renewcommand{\liminf}[1]{\lim_{#1 \ra \infty}} % Excluded when using liminf
\newcommand{\limninf}[1]{\lim_{#1 \ra -\infty}}
...
\end{Verbatim}
만약 어떤 문서에서 \Verb+\usepackage{woosung_math}+ 를 한 다음 아래와 같은 수식을
입력하면
\begin{Verbatim}[tabsize=4,xleftmargin=2em]
\forall n \in \N, \liminf{c} cn = \infty
\end{Verbatim}
아래와 동일한 효과를 얻습니다.
\begin{Verbatim}[tabsize=4,xleftmargin=2em]
\forall n \in \mathbb N, \lim_{n \rightarrow \infty} cn = \infty
\end{Verbatim}
즉, 이런 내용의 수식이 입력됩니다.
\begin{align*}
	\forall n \in \N, \liminf{c} cn = \infty
\end{align*}
이와 같이 좀 더 단순하게 수학 기호를 입력할 수 있습니다. 대표적인 것들을 몇 개 꼽자면,
\begin{itemize}
	\item \Verb+\ra+: \Verb+\rightarrow+ ($\ra$)
	\item \Verb+\la+: \Verb+\leftarrow+ ($\la$)
	\item \Verb+\Ra+: \Verb+\rightarrow+ ($\Ra$)
	\item \Verb+\La+: \Verb+\leftarrow+ ($\La$)
	\item \Verb+\R, \N, \Z+: ($\R, \N, \Z$)
	\item \Verb+\ep, \dt+: ($\ep, \dt$)
	\item \Verb+\inv+: \Verb+^{-1}+ ($\inv$)
	\item \Verb+\ie+: (\ie)
	\item \Verb+\eg+: (\eg)
\end{itemize}
가 있습니다. 또한, \Verb+\numberthis+ 라는 명령어도 있습니다. 이는
\Verb+\begin{align*} ... \end{align*}+ 환경 내에서 일부 수식에만 번호를 매길 때 사용합니다.
예를 들어 아래와 같이 입력하면
\begin{Verbatim}[tabsize=4,xleftmargin=2em]
\begin{align*}
	\sum_{i=1}^{n} i &= 1 + 2 + 3 + \cdots + n \\
		&= \frac{n(n+1)}{2} \numberthis
\end{align*}
\end{Verbatim}
이와 같은 수식이 나옵니다.
\begin{align*}
	\sum_{i=1}^{n} i &= 1 + 2 + 3 + \cdots + n \\
		&= \frac{n(n+1)}{2} \numberthis
\end{align*}
오직 두 번째 줄에서만 \Verb+\numberthis+ 처리를 해줬기 때문에 수식 번호가 매겨진 것입니다.
이는 \\ \Verb+\usepackage{woosung_math}+ 를 넣어줘야 사용할 수 있습니다.


\section{.latexrc 단축키 설명}
자주 사용하는 명령어는 \textbf{볼드체}로 강조해놓았습니다.

\subsection*{컴파일 및 pdf 파일 열기}
\begin{itemize}
	\item \textbf{[F3] 버튼}: \texttt{pdflatex} 컴파일러로 해당 \texttt{tex} 파일을
		컴파일합니다. 컴파일 로그는 \texttt{/tmp/.o.out} 위치에 저장합니다.
	\item \textbf{[F4] 버튼}: 컴파일된 \texttt{pdf} 파일을 default 뷰어로 엽니다.
	\begin{itemize}
		\item 윈도우에서 bash를 이용해 작업하시는 분은 아마 안 될겁니다.
			여기에 대한 메뉴얼은 추후 업데이트하겠습니다.
	\end{itemize}
	\item \textit{[Ctrl+F3] 버튼}: gnome-terminal 에서 컴파일 로그를 출력하면서 컴파일합니다.
		[F3]으로 컴파일해서 오류가 생겼을 때 사용하기를 권장드립니다.
	\item \textit{[F5] 버튼}: 오류 로그를 화면 오른쪽에 탭으로 띄웁니다.
\end{itemize}


\subsection*{커서의 이동}
`\Verb+<<-->>+' 무늬를 찾고 다음 위치로 이동하는 작업으로 편리하게 커서를 이동시킬 수
있습니다.  예를 들어 작업하고 있는 \texttt{tex} 파일이 다음과 같다고 가정해봅시다.
커서의 위치는 `\Verb+|+' 표시로 나타내겠습니다.
\begin{Verbatim}[tabsize=4,xleftmargin=2em]
$$ sum_{i=0|}^{<<-->>} <<-->> $$
\end{Verbatim}
여기서 [Ctrl+j] 버튼을 누르면 다음과 같이 커서가 이동합니다.
\begin{Verbatim}[tabsize=4,xleftmargin=2em]
$$ sum_{i=0}^{|} <<-->> $$
\end{Verbatim}
`\texttt{n}'을 입력한 뒤 [Ctrl+j] 버튼을 누르면 커서가 다음과 같이 이동합니다.
\begin{Verbatim}[tabsize=4,xleftmargin=2em]
$$ sum_{i=0}^{n} | $$
\end{Verbatim}
이와 같은 기능으로 키보드만으로도 정말 빠르고 편리하게 조판작업을 할 수 있습니다.
\begin{itemize}
	\item \textbf{[Ctrl+j] 버튼}: 다음 `\Verb+<<-->>+' 표시로 이동하고 그 문양을 지웁니다.
	\item \textbf{[Ctrl+k] 버튼}: 커서의 위치에 `\Verb+<<-->>+' 문양을 새로 입력합니다.
\end{itemize}


\subsection*{문서 정보 입력}
\begin{itemize}
	\item \textbf{\bs Document 입력}: \Verb+\begin{document} ... \end{document}+ 텍스트를
		입력합니다.
	\item \textbf{\bs Info 입력}: \Verb+\title{} \author{} \date{}+ 텍스트를 입력합니다.
	\item \textit{\bs MoreInfo 입력}: \Verb+\title{} \subtitle \author{} \date{} \institute+
		텍스트를 입력합니다. 오직 \texttt{beamer} 클래스에서만 적용됩니다.
\end{itemize}


\subsection*{패키지 불러오기 및 기타}
\begin{itemize}
	\item \textbf{\bs Package 입력}: \Verb+\usepackage{}+ 텍스트를 입력합니다.
		패키지를 불러올 때 사용합니다. 사용 예시:
\begin{Verbatim}[tabsize=4,xleftmargin=2em]
\usepackage{kotex}      % Hangul
\usepackage{woosung_hw} % Woosung's style file
\end{Verbatim}
	\item \textit{\bs Newcomm 입력}: \Verb+\newcommand{}{<<-->>}+ 텍스트를 입력합니다.
	\item \textit{\bs Renewcomm 입력}: \Verb+\renewcommand{}{<<-->>}+ 텍스트를 입력합니다.
	\item[] \textit{(environment는 .sty 파일에서 미리 처리해놓는 경우가 많기 때문에
		따로 단축키로 넣지 않았습니다.)}
\end{itemize}


\subsection*{챕터와 섹션}
\begin{itemize}
	\item \textbf{\bs Chap 입력}: \Verb+\chapter{}+ 텍스트를 입력합니다.
	\item \textbf{\bs Sec 입력}: \Verb+\section{}+ 텍스트를 입력합니다.
	\item \textit{\bs SSec 입력}: \Verb+\subsection{}+ 텍스트를 입력합니다.
	\item \textit{\bs SSSec 입력}: \Verb+\subsubsection{}+ 텍스트를 입력합니다.
	\item \textbf{\bs Sec* 입력}: \Verb+\section*{}+ 텍스트를 입력합니다.
	\begin{itemize}
		\item \bs Chap*, \bs SSec*, \bs SSSec* 도 유사합니다.
		\item 챕터 및 섹션에 번호매김을 하지 않습니다.
	\end{itemize}
\end{itemize}


\subsection*{텍스트 강조}
\begin{itemize}
	\item \textbf{\bs tbf 입력}: 볼드체 텍스트 입력 레이아웃을 적습니다. \textbf{(Text)}
	\item \textbf{\bs tit 입력}: 기울임 텍스트 입력 레이아웃을 적습니다. \textit{(Text)}
	\item \textbf{\bs ttt 입력}: 모노스페이스 텍스트 입력 레이아웃을 적습니다. \texttt{(Text)}
	\item \textit{\bs tsc 입력}: Scientific 텍스트 입력 레이아웃을 적습니다. \textsc{(Text)}
	\item \textit{\bs txt 입력}: 플레인 텍스트 입력 레이아웃을 적습니다. 수식 모드에서
		플레인 텍스트로 적을 때 주로 사용됩니다. (\Verb+\text{<<-->>}<<-->>+)
\end{itemize}


\subsection*{번호 매김}
\begin{itemize}
	\item \textbf{\bs Enumerate 입력}: 1) 2) 3) ... 순의 번호매김 레이아웃을 적습니다.
	\begin{itemize}
		\item \textit{\bs Enumeratei 입력}: i) ii) iii) ... 순
		\item \textit{\bs Enumeratea 입력}: a) b) c) ... 순
		\item \textit{\bs EnumerateI 입력}: \textbf{i) ii) iii)} ... 순 (볼드체)
		\item \textit{\bs EnumerateA 입력}: \textbf{a) b) c)} ... 순 (볼드체)
	\end{itemize}
	\item \textbf{\bs Itemize 입력}: 번호 없이 리스팅하는 레이아웃을 적습니다.
\end{itemize}

\subsection*{그림, 표, 가운데 정렬}
\begin{itemize}
	\item \textbf{\bs Figure 입력}: 그림을 넣는 레이아웃을 적습니다.
	\begin{itemize}
		\item 아래와 같이 입력이 됩니다. 
\begin{Verbatim}[tabsize=4,xleftmargin=2em]
\begin{figure}[!htb]
	\centering
	\includegraphics[width=.|\textwidth]{<<-->>}
	\caption{<<-->>}
\end{figure}<<-->>
\end{Verbatim}
		\item[] 올바른 입력 예시는 다음과 같습니다.
\begin{Verbatim}[tabsize=4,xleftmargin=2em]
\begin{figure}[!htb]
	\centering
	\includegraphics[width=.8\textwidth]{src/woosung.png}
	\caption{Photo of Woosung Song}
\end{figure}
\end{Verbatim}
		\item 커서 이동인 [Ctrl+j] 버튼을 이용하여 차례대로 채우시면 됩니다.
	\end{itemize}
	\item \textbf{\bs Table 입력}: 표를 넣는 레이아웃을 적습니다.
	\item \textit{\bs Tabular 입력}: 표와 함께 캡션을 정의할 수 있는 레이아웃을 적습니다.
		라벨링이 가능합니다.
\end{itemize}


\subsection*{수학 관련 단축키}
\subsubsection*{Align 및 Array}
\begin{itemize}
	\item \textbf{\bs Align 입력}: \Verb+\begin{align*} ... \end{align*}+ 를 입력합니다.
		한 번에 여러 줄의 수식을 입력할 때 사용합니다. 대부분의 가운데 정렬 수식을
		적을때 이 코드를 사용합니다.
	\begin{itemize}
		\item Align 을 이용해 한 줄만 입력하면 \Verb+$$ .. $$+와 거의 동일한 효과를 냅니다.
\begin{Verbatim}[tabsize=4,xleftmargin=2em]
\begin{align*}
	\sum_{i=1}^{n} i = \frac{n(n+1)}{2}
\end{align*}
\end{Verbatim}
			\begin{align*}
				\sum_{i=1}^{n} i = \frac{n(n+1)}{2}
			\end{align*}
		\item 여러 줄을 줄맞춤하고 싶은 경우 \& 기호를 이용하면 됩니다. 줄구분은
			\Verb+\\+ 를 이용하세요.
\begin{Verbatim}[tabsize=4,xleftmargin=2em]
\begin{align*}
	\sum_{i=1}^{n} i &= 1 + 2 + 3 + \cdots + n \\
		&= \frac{n(n+1)}{2}
\end{align*}
\end{Verbatim}
			\begin{align*}
				\sum_{i=1}^{n} i &= 1 + 2 + 3 + \cdots + n \\
					&= \frac{n(n+1)}{2}
			\end{align*}
		\item 오른쪽에 주석을 달고 싶은 경우 \& 기호를 같은 줄에 하나 더 추가하면 됩니다.
\begin{Verbatim}[tabsize=4,xleftmargin=2em]
\begin{align*}
	\sum_{i=1}^{n} i &= 1 + 2 + 3 + \cdots + n \\
		&= \frac{n(n+1)}{2} & \text{By the mathematical induction}
\end{align*}
\end{Verbatim}
			\begin{align*}
				\sum_{i=1}^{n} i &= 1 + 2 + 3 + \cdots + n \\
					&= \frac{n(n+1)}{2} & \text{By the mathematical induction}
			\end{align*}
		\item 수식에 번호를 달고 싶으면 그 줄 끝에 \Verb+\numberthis+ 를 다세요.
\begin{Verbatim}[tabsize=4,xleftmargin=2em]
\begin{align*}
	\sum_{i=1}^{n} i &= 1 + 2 + 3 + \cdots + n \numberthis \\
		&= \frac{n(n+1)}{2} & \text{By the mathematical induction}
\end{align*}
\end{Verbatim}
			\begin{align*}
				\sum_{i=1}^{n} i &= 1 + 2 + 3 + \cdots + n \numberthis \\
					&= \frac{n(n+1)}{2} & \text{By the mathematical induction}
			\end{align*}
	\end{itemize}
	\item \textit{\bs Array 입력}: 수식 모드 내에서 표 모양 데이터를 입력할 때
		사용합니다.
\end{itemize}

\subsubsection*{정의, 정리 선언 (woosung\_hw.sty 파일 불러올 때만 유효)}
\begin{itemize}
	\item \textbf{\bs Thm 입력}: \Verb+\begin{theorem} ... \end{theorem}+ 을 입력합니다.
	\item \textbf{\bs Def 입력}: \Verb+\begin{definition} ... \end{definition}+ 을 입력합니다.
	\item \textbf{\bs Proof 입력}: \Verb+\begin{proof} ... \end{proof}+ 을 입력합니다.
	\begin{itemize}
		\item 기타 \textit{\bs Lem, \bs Prop, \bs Coro, \bs Ex, \bs Prob, \bs Remark 입력}:
			각각 lemma, proposition, corollary, example, problem, remark를 의미합니다.
		\item 문서 작성 예시:
\begin{Verbatim}[tabsize=4,xleftmargin=2em]
\begin{theorem}
	Let $X$ be a metrizable space and $A \subset X$.
	If $x \in \Cl A$, then there is a sequence $x_n$ on $A$ converges to $x$.
\end{theorem} 
\begin{proof}
	Assume $x \in \Cl A$.
	Note that $B(x, 1 / n)$ is an open neighborhood of $x$
	for any $n \in \N$, and it intersects $A$.
\end{proof}
\end{Verbatim}
			\begin{theorem}
				Let $X$ be a metrizable space and $A \subset X$.
				If $x \in \Cl A$, then there is a sequence $x_n$ on $A$ converges to $x$.
			\end{theorem} 
			\begin{proof}
				Assume $x \in \Cl A$.
				Note that $B(x, 1 / n)$ is an open neighborhood of $x$
				for any $n \in \N$, and it intersects $A$.
			\end{proof}
	\end{itemize}
\end{itemize}

\subsubsection*{괄호 입력}
\begin{itemize}
	\item \textbf{\bs Brace$($ 입력}: 소괄호를 좌 우 입력해주고 커서로 쉽게 이동할 수 있도록
		레이아웃을 구성해줍니다.
	\item \textbf{\bs Brace$\{$ 입력}: 중괄호 입력, 레이아웃 구성해줍니다. 단, \Verb+\{ .. \}+
		형태로 구성합니다.
	\item \textbf{\bs Brace$[$ 입력}: 대괄호 입력, 레이아웃 구성해줍니다.
	\item \textit{\bs Brace$.$ 입력}: 왼쪽은 중괄호를, 오른쪽은 온점을 입력합니다.
		아래와 같은 수식을 만들 때 array와 함께 주로 사용됩니다.
\begin{Verbatim}[tabsize=4,xleftmargin=2em]
\begin{align*}
	X &= \left\{
		\begin{array}{ll}
			f(x) & \text{if $x \geq 0$} \\
			g(x) & \text{otherwise}
		\end{array}
		\right.
\end{align*}
\end{Verbatim}
\begin{align*}
	X &= \left\{
		\begin{array}{ll}
			f(x) & \text{if $x \geq 0$} \\
			g(x) & \text{otherwise}
		\end{array}
		\right.
\end{align*}
	\item \textit{\bs Bracec 입력}: ceil 함수 기호에 대응되는 괄호를 양 옆에 입력합니다.
	\item \textit{\bs Bracef 입력}: floor 함수 기호에 대응되는 괄호를 양 옆에 입력합니다.
\end{itemize}

\subsubsection*{기타 수학기호 단축키}
\begin{itemize}
	\item \textbf{\bs frac$\{$ 입력}: 분수 입력 레이아웃을 완성합니다.
		\Verb+\frac{|}{<<-->>} <<-->>+
	\item \textbf{\bs sum\_ 입력}: 합 입력 레이아웃을 완성합니다.
		\Verb+\sum_{|}^{<<-->>} <<-->>+
	\item \textbf{\bs prod\_ 입력}: 곱 입력 레이아웃을 완성합니다.
		\Verb+\prod_{|}^{<<-->>} <<-->>+
	\item \textbf{\bs lim\_ 입력}: 극한 입력 레이아웃을 완성합니다.
		\Verb+\lim_{|} <<-->>+
	\item \textit{\bs bigcup\_ 입력}: 합집합 입력 레이아웃을 완성합니다.
		\Verb+\bigcup_{|} <<-->>+
	\item \textit{\bs bigcap\_ 입력}: 교집합 입력 레이아웃을 완성합니다.
		\Verb+\bigcap_{|} <<-->>+
	\item \textit{\bs bigcup\char`\^ 입력}: 합집합 입력 레이아웃을 완성합니다.
		\Verb+\bigcup_{|}^{<<-->>} <<-->>+
	\item \textit{\bs bigcap\char`\^ 입력}: 교집합 입력 레이아웃을 완성합니다.
		\Verb+\bigcap_{|}^{<<-->>} <<-->>+
\end{itemize}


\subsection*{코드 입력 (모노스페이스)}
\begin{itemize}
	\item \textbf{\bs Verbatim 입력}: 코드 입력 레이아웃을 완성합니다.
	\begin{itemize}
		\item 예를 들어 다음과 같이 코드를 채우면 (백슬래시가 띄어진건 컴파일이 안
			돼서 그런거라 이해해주세요..)
\begin{Verbatim}[tabsize=4,xleftmargin=2em]
\begin{Verbatim}[tabsize=4,xleftmargin=2em]
def woosung():
	print("Hello World!")
	print("I like you");
\ end{Verbatim}
\end{Verbatim}
		\item[] 이런 결과가 나옵니다.
\begin{Verbatim}[tabsize=4,xleftmargin=2em]
def woosung():
	print("Hello World!")
	print("I like you");
\end{Verbatim}
	\end{itemize}
	\item \textit{\bs NVerbatim 입력}: 줄 번호도 나오는 코드 입력 레이아웃을 완성합니다.
\begin{Verbatim}[numbers=left,tabsize=4,xleftmargin=2em]
def woosung():
	print("Hello World!")
	print("I like you");
\end{Verbatim}
\end{itemize}


\subsection*{Beamer 설정}
LaTeX으로 강의 슬라이드를 만들거나 학회용 포스터를 사용할 때 주로 사용하는 클래스가
\texttt{beamer} 입니다. 관련 세팅에 대한 메뉴얼은 추후 추가하도록 하겠습니다.

\end{document}
